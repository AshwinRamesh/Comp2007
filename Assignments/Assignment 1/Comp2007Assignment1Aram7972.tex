\documentclass{article}
\usepackage[utf8]{inputenc}
\usepackage{amsmath}
\usepackage[pdftex]{graphicx}
\usepackage{amsfonts}
\usepackage{amssymb}
\usepackage[hmargin=2.5cm,vmargin=0.5cm]{geometry}
\author{Ashwin Ramesh}
\title{Comp2007 - Assignment 1}
\begin{document}
\maketitle

\section{Problem 1}

\subsection{Overview of Problem 1}

Polynomial calculation can be a very expensive computation to run and therefore efficiency is integral when creating algorithms to solve this problem. This problem introduces two algorithms. Algorithm 1 uses a naive approach, recalculating the new power of x on every iteration. On the other hand, Algorithm 2 uses Horner's rule by calculating the new power by using the previous power of x. Below is the asymptotically tight analysis of both algorithms. 

\subsection{Algorithm Analysis}


\begin{flushleft}

\textbf{Upper Bounds:} $T(n)$ is $O(f(n))$ if there exists constants $c > 0$ and $n_{0} \geqq 0$ such that for all $n \geqq n_{0}$, we have $T(n) \leqq c*f(n)$.
\end{flushleft}

\begin{flushleft}
\textbf{Lower Bounds:} $T(n)$ is $\Omega (f(n))$ if there exists constants $c > 0$ and $n_{0} \geqq 0$ such that for all $n \geqq n_{0}$, we have $T(n) \geqq c*f(n)$.
\end{flushleft}

\begin{flushleft}
\textbf{Tight Bounds:} $T(n)$ is $\Theta (f(n))$ if $ T(n) $ is both $O(f(n))$ and $\Omega (f(n))$.
\end{flushleft}

\subsection{Algorithm 1 - NAIVE(x,A) Analysis}
\begin{itemize}
\item Line 3: $O(n)$ time as it is an iteration from 0 to n

\item Line 4: $O(1)$ time

\item Line 5: $O(n)$ time as i varies from 1 to n

\item Line 6/7: $O(1)$ time
\end{itemize}


\textbf{Upper Bound Time Complexity of Naive Approach: $O(n) * O(n) = O(n^2)$}

\mbox{}

\textbf{Lower Bound Time Complexity of Naive Approach:} similarly the lower bound will be $ \Omega(n^2)$

\mbox{}

This means that the \textbf{Tight Bound Complexity} will be $ \Theta(n^2)$

\subsection{Algorithm 2 - HORNER(x,A) Analysis}
\begin{itemize}
\item Line 3: $O(n)$ time as it is an iteration from 0 to n

\item Line 4: $O(1)$ as it uses the result from above to calculate addition

\end{itemize}


\textbf{Upper Bound Time Complexity of Naive Approach: $O(n) * O(1) = O(n)$}

\mbox{}

\textbf{Lower Bound Time Complexity of Naive Approach:} similarly the lower bound will be $ \Omega(n)$

\mbox{}

This means that the \textbf{Tight Bound Complexity} will be $ \Theta(n)$

\end{document}